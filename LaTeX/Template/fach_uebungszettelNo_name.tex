\input{src/header}											% bindet Header ein (WICHTIG)

\newcommand{\dozent}{Claudia Müller-Birn und Barry Linnert}					% <-- Names des Dozenten eintragen
\newcommand{\tutor}{Florian Alex}						% <-- Name eurer Tutoriun eintragen
\newcommand{\tutoriumNo}{Übung 05}				% <-- Nummer im KVV nachschauen
\newcommand{\ubungNo}{1}									% <-- Nummer des Übungszettels
\newcommand{\veranstaltung}{Nichtsequentielle und verteilte Programmierung}	% <-- Name der Lehrveranstaltung eintragen
\newcommand{\semester}{SoSe 2020}						% <-- z.B. SoSo 17, WiSe 17/18
\newcommand{\studenten}{Qianli Wang und Nazar Sopiha}			% <-- Hier eure Namen eintragen

% /////////////////////// BEGIN DOKUMENT /////////////////////////
\begin{document}
\input{src/titlepage}										% erstellt die Titelseite


% /////////////////////// Aufgabe 1 /////////////////////////
\section{Aufgabe: Quellcode einbinden \hfill (5 Punkte)}
{\itshape Aufgabenstellung des Dozenten}

\lstinputlisting[style=py,									% Style
	caption={Beispiel: Einbindung von Programmcode},		% Beschriftung
	firstnumber={1},										% Start der Nummerierung
	firstline={1},											% 1. Codezeile
	lastline={7}]											% letzte Codezeile
{src/beispielprogramm.py}

% ///////////////////// Aufgabe 2 ///////////////////
\section{Aufgabe: Mathemodus \hfill (5 Punkte)}
\begin{enumerate}[a)]
\item {\itshape Welche nützlichen Befehle gibt es?}		% Aufgabenteil a)
	\begin{itemize}\setlength{\itemsep}{-1ex}				% verringert den Abstand zwischen den Items
	\item Mengen: $\N, \Z, \Q, \R$
	\item Widerspruch: $\Bolt$
	\item $\zz$, \gdw, \oBdA
	\item runden: $\floor{\frac{n}{2}}, \ceil{\frac{n}{3}}$
	\item Funktion: $f \from \N \to \N$
	\item Komplexitätsklassen: $\P, \NP$
	\item $\Landau$-Notation
	\item Wahrheitswerte: $\True, \False$
	%\item Währung: \Eu
	\end{itemize}

\item {\itshape Fallunterscheidung}						% Aufgabenteil b)
	\[ \text{collatz}(x) = \begin{cases}
			         \floor{\frac{n}{2}}, & \text{falls } n \mod 2 = 0 \\
			         3\cdot n+1, & \text{sonst} \\
	                \end{cases}
	\]
	
\item {\itshape Matrix}										% Aufgabenteil c)
	\[ A = \begin{pmatrix}
		1 & 2 & 3 \\
		4 & 5 & 6 \\
	\end{pmatrix}
	\]
\end{enumerate}

% ///////////////////// Aufgabe 3 ///////////////////
\section{Aufgabe: Zeichnen mit tikz I \hfill (5 Punkte)}
{\itshape Erstellen Sie einen Hyperlink und zeichnen Sie einen Pfeil mittels tikz}

Tikzmark zum nachträglichen Einfügen von Elementen:
% malt ein Viereck mit runden Ecken
\begin{tikzpicture}[remember picture, overlay] % stets 2x compilieren
\draw[draw=FUgreen,line width=1.5pt, rounded corners, fill=FUblue!10!]
([shift={(-0.2em,2.2ex)}] pic cs:m3) rectangle
([shift={(0.2em,-0.6ex)}]pic  cs:m2);
\end{tikzpicture}

\begin{itemize}
\item Um diesen Text zu verstehen, muss man\tikzmark{m1}
\item zum vorherigen Punkt \tikzmark{m3}zurück kehren\tikzmark{m2}
\end{itemize}

% malt einen Pfeil
\begin{tikzpicture}[remember picture, overlay] % stets 2x compilieren
\draw[line width=1pt, <-, >=stealth]
(pic cs:m1)
	to[bend left]
	node[right, inner sep=1em]{Beschriftung}
(pic  cs:m2);
\end{tikzpicture}

\section{Aufgabe: Zeichnen mit tikz II (10 Punkte)}
{\itshape Zeichnen Sie einen Automaten}

\begin{figure}[H]% float-Paket benötigt
\centering						% zentriert Figur
\begin{tikzpicture}[>=stealth',
	shorten >=1pt,
    auto,
	node distance=5em,				% Abstand der Knoten zueinander
    scale = 1,
    transform shape]
	% Quelle: http://tex.stackexchange.com/questions/45734/drawing-graphs-in-latex
	% erster Strang
    \node[initial,state]   (A)              {$q_0$}; % Startknoten
    \node[state]           (B) [above right of=A] {$q_1$};
    \node[state,accepting] (C) [below right of=A] {$q_2$};

	\path[->] % zeichnet die Pfeile von (A) nach (C)
      (A) edge                 node         {$1$} (B)
      (A) edge                 node [left] {$0$} (C)
      (B) edge [loop right]    node         {$0$} (B)
      (B) edge                 node         {$1$} (C)
      (C) edge [bend right=45] node [right] {$1$} (B)
      (C) edge [loop right]    node         {$0$} (C);
	% (startknoten)
		% Option: "edge [bend right=45]" macht Kurve um 45° nach rechts
		% Option "edge [loop]" macht rekursiven Pfeil
		% "node [above, below, right, left, sloped, align=center, pos=0.5] {text}" für Beschriftung
		% (zielknoten)
		% sloped dreht text zur pfeilrichtung
		% right, left richtet text rechts bzw. links vom Pfeil aus (nicht ganz eindeutig)
		% pos aus [0,1] verschiebt text entlang der Pfeilrichtung, 0.5 ist mittig
		% above, below richtet text ober- bzw. unterhalb des Pfeils aus
		% align legt Ausrichtung des texts fest
\end{tikzpicture}
\end{figure}

% /////////////////////// END DOKUMENT /////////////////////////
\end{document}